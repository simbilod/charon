%
% $Id: SANDExampleArticleNotstrict.tex,v 1.26 2009-05-01 20:59:19 rolf Exp $
%
% This is an example LaTeX file which uses the SANDreport class file.
% It shows how a SAND report should be formatted, what sections and
% elements it should contain, and how to use the SANDreport class.
% It uses the LaTeX article class, but not the strict option.
% It uses jpeg logos and files to show how pdflatex can be used
%
% Get the latest version of the class file and more at
%    http://www.cs.sandia.gov/~rolf/SANDreport
%
% This file and the SANDreport.cls file are based on information
% contained in "Guide to Preparing {SAND} Reports", Sand98-0730, edited
% by Tamara K. Locke, and the newer "Guide to Preparing SAND Reports and
% Other Communication Products", SAND2002-2068P.
% Please send corrections and suggestions for improvements to
% Rolf Riesen, Org. 9223, MS 1110, rolf@cs.sandia.gov
%
\documentclass[12pt]{SANDreport}
\usepackage{pslatex}
\usepackage{amsmath}
\usepackage{amssymb}
\usepackage{mathptmx}	% Use the Postscript Times font
\usepackage[FIGBOTCAP,normal,bf,tight]{subfigure}
\usepackage{afterpage}
\usepackage{fancyvrb}
\usepackage{makeidx,amsfonts,longtable,array}
%\usepackage[colorlinks=true,linkcolor=red!80!black,urlcolor=magenta!80!black]{hyperref}
\usepackage{booktabs}
\usepackage{tabularx}
\usepackage{siunitx}
\usepackage{xfrac}
\usepackage[printonlyused]{acronym}
\usepackage{bpchem}
\usepackage{dcolumn}
\usepackage{listings}
\lstset{
basicstyle=\small\ttfamily,
columns=flexible,
breaklines=true
}
\usepackage{color}
\usepackage{dirtree}
\usepackage{hhline}
\usepackage{commath}
\usepackage{physics}
\usepackage[update,prepend]{epstopdf} 
\usepackage{url}
\usepackage{verbatim}
%\usepackage{draftwatermark}
%\SetWatermarkScale{.5}
%\SetWatermarkText{Sample, contains no OUO}
%
% Include this file, if your SANDIA Report is Official Use Only.
% This is an example provided by Michael Kaneshige, and you'll have to
% change the wording and data according to your needs.
% See Appendix A in "Guide to Preparing SAND Reports and Other
% Communication Products". pages 55ff.
%
\SANDreleaseType{Official Use Only $\bullet$ Export Controlled Information}
\SANDmarkTopBottom{\CoverFont{b}{12}{10pt}OFFICIAL USE ONLY}

\SANDmarkCover{
  \framebox{
    \begin{minipage}{3.6in}
        \vspace{.1in}
	\begin{center}\textbf{OFFICIAL USE ONLY}\end{center}
	May be exempt from public release under the Freedom of 
	Information Act (5 U.S.C. 552), exemption number and 
        category: 3. Statutory Exemption.  \\
	\\
        Department of Energy review required before public release \\
	\\
	Name/Org:  \underline{Lawrence C Musson~/~01355} Date: \underline{September 20, 2015}\\ \\
	Guidance (if applicable): \\
    \end{minipage}
  } \\ \\

  \begin{minipage}{3.75in} \textbf{EXPORT CONTROLLED INFORMATION}\\ \\
    {\bf WARNING}--This document contains technical data whose export
    is restricted by the Atomic Energy Act of 1954, as amended
    42. U.S.C. \S 2011. {\em et seq.} Violations of these export
    laws are subject to severe criminal penalties.\\

    Further dissemination authorized to the Department of Energy and
    DOE contractors only; other requests shall be approved by the
    originating facility or higher DOE programmatic authority. \\
    \end{minipage} }

%

\SANDreviewedBy{
\begin{minipage}{4in}
\noindent{\bf{Reviewed by:}}  Gary L Hennigan 
$\_\_\_\_\_\_\_\_\_\_\_\_\_\_\_\_\_\_\_\_\_\_\_\_\_\_\_\_\_\_\_\_\_\_\_\_\_\_\_\_\_\_\_\_\_\_\_\_$ \\
\noindent{\bf Date:} 6 October 2015
\end{minipage}
}



% If you want to relax some of the SAND98-0730 requirements, use the "relax"
% option. It adds spaces and boldface in the table of contents, and does not
% force the page layout sizes.
% e.g. \documentclass[relax,12pt]{SANDreport}
%
% You can also use the "strict" option, which applies even more of the
% SAND98-0730 guidelines. It gets rid of section numbers which are often
% useful; e.g. \documentclass[strict]{SANDreport}

%    \makeindex


% ---------------------------------------------------------------------------- %
%
% Set the title, author, and date
%
    \title{Charon Generate Parser Scripts and Their Use}

    \author{Lawrence C Musson \\
	  Electrical Models \& Simulation \\
	  Sandia National Laboratories\\
	  P.O. Box 5800\\
	  Albuquerque, NM 87185-1177 \\
	  lcmusso@sandia.gov \\
	 }

    % There is a "Printed" date on the title page of a SAND report, so
    % the generic \date should generally be empty.
    \date{}


% ---------------------------------------------------------------------------- %
% Set some things we need for SAND reports. These are mandatory
%
\SANDnum{SAND2019-xxxx}
\SANDprintDate{February 2019}
\SANDauthor{Lawrence C Musson}


% ---------------------------------------------------------------------------- %
% Include the markings required for your SAND report. The default is "Unlimited
% Release". You may have to edit the file included here, or create your own
% (see the examples provided).
%
% %
% Inlcude this file, if your SANDIA Report is Unlimited Release.
% This is not really necessary, since Unlimited Release is the default.
%
\SANDreleaseType{Unlimited Release}
\SANDmarkCover{Approved for public release; further dissemination unlimited.}
 % Not needed for unlimted release reports



% ---------------------------------------------------------------------------- %
% The following definition does not have a default value and will not
% print anything, if not defined
%
%\SANDsupersed{SAND1901-0001}{January 1901}


% ---------------------------------------------------------------------------- %
%
% Start the document
%
\begin{document}
    \maketitle

    % ------------------------------------------------------------------------ %
    % An Abstract is required for SAND reports
    %
    \begin{abstract}
	
Larry writes this



    \end{abstract}


    % ------------------------------------------------------------------------ %
    % An Acknowledgement section is optional but important, if someone made
    % contributions or helped beyond the normal part of a work assignment.
    % Use \section* since we don't want it in the table of context
    %
    \clearpage
    \section*{Acknowledgment}
	    Thanks to Larry for being such an awesome dude.




    % ------------------------------------------------------------------------ %
    % The table of contents and list of figures and tables
    % Comment out \listoffigures and \listoftables if there are no
    % figures or tables. Make sure this starts on an odd numbered page
    %
    \cleardoublepage		% TOC needs to start on an odd page
    \tableofcontents
    \listoffigures
    \listoftables


    % ---------------------------------------------------------------------- %
    % An optional preface or Foreword
    %\clearpage
    %\section*{Preface}
    %\addcontentsline{toc}{section}{Preface}
    %	\input{CommonPreface}


    % ---------------------------------------------------------------------- %
    % An optional executive summary
    %\clearpage
    %\section*{Summary}
    %\addcontentsline{toc}{section}{Summary}
	%\input{CommonSummary}


    % ---------------------------------------------------------------------- %
    % An optional glossary. We don't want it to be numbered
    %\clearpage
    %\section*{Nomenclature}
    %\addcontentsline{toc}{section}{Nomenclature}
    %\begin{description}
	%\item[dry spell]
	 %   using a dry erase marker to spell words
	%\item[dry wall]
	 %   the writing on the wall
	%\item[dry humor]
	 %   when people just do not understand
	%\item[DRY]
	 %   Don't Repeat Yourself
    %\end{description}


    % ---------------------------------------------------------------------- %
    % This is where the body of the report begins; usually with an Introduction
    %
    \SANDmain		% Start the main part of the report

    \label{Intro}
    %\input{Charon_generateParsersIntroduction}


    



\section{Charon's GenerateInterpreter script}

The Charon interpreter is a Python script that is intended to simplify
the use of Charon by providing a simpler, easier to read syntax than
creating xml or yaml parameter lists.  One of the most complex tasks
the script must undertake is to parse the user's input and map it to
the parameter lists.  Or rather, one of the most complex tasks for the
developer to undertake is to write the parsers that provide the
mapping.  This can be particularly challenging for developers who are
developing routinely in C++, but rarely or never in Python.  

The Charon generateInterpreter script is a script that writes scripts.
Namely, it will take a simplified input provided by a developer who is
required to generate new input to Charon based on a new feature or
otherwise modified capability of the software and write parsers that
will interpret and map user input to parameter list specifications.
The Charon interpreter file can contain single lines of input or
multiple lines of input collected into nested blocks.  The block input
is general and so nesting can go arbitrarily deep.  In principle,
however, the depth of nesting should be kept shallow and replaced with
creativity in the input syntax.  This avoids confusion for users.

The following sections describe the input file syntax for the parser
generators for both line inputs and block inputs.

\subsection{Charon Create Line Parser}

The most fundamental element of the Charon interpreter is the line of
input regardless of whether that line exists in isolation or is a part
of a larger block of input.  In either case, the line of interpreter
input must be defined by its own parser.  The parser takes the line of
input and will map it to, usually, many lines of XML or YAML input in
the parameter lists used by Charon.  The createLineParser script in
Charon's generateInterpreter scripts takes input that ultimately
defines the interpreter line, all of its arguments and options
including user help content and maps it to parameter list content that
Charon and it's libraries use.  This section describes the input syntax
that creates the parser for a single line of input regardless of
whether it's part of a block or not.

The generateInterpreter script that creates the parser will read input
files from an immediate subdirectory called parseInputs.  That
directory will, in general, contain inputs for all of the
interpreter's parsers.  If the script is called without arguments, it
will sequentially process every input file in the subdirectory.
Alternatively, the name of the input file can be specified as a
command line argument to the script and only that input file will be
processed.  All of the line parser input files {\bf{must}} contain a
.inp suffix.

The create line parser input file contains several lines that
ultimately define the parser.  In general, the input file is case
insensitive except in certain instances where the parameter list
information is defined.  Parameter lists are case sensitive.
Moreover, the input file lines can be included in any order in the
file, but they will be presented here in what the author considers to
be a logical order.

The input to the parser generator can be grouped into two larger
categories: the lines that define the interpreter input, and the lines
that define the parameter list content generated by the parser.  The
parser that gets created is actually a Python class which the
interpreter will instantiate into an object that provides mapping
between user input to the interpreter and the parameter list output.
The interpreter group is presented first.

The first order of business is to name the new Python class that
defines the parser.  It is accomplished by the line:
\begin{lstlisting}
interpreter name NewParser
\end{lstlisting}
This line tells the generator to create a new Python class
called \begin{lstlisting}charonLineParserNewParser\end{lstlisting} in a
  file
  named \begin{lstlisting}charonLineParserNewParser.py\end{lstlisting} in
  the adjacent parsers subdirectory.  In this instance,
  ``interpreter'' and ``name'' are case insensitive whereas
  ``NewParser'' is not.  Case if the class and file names follow the
  input.

The line with the specifiers ``interpreter inputLine'' creates the
input syntax that the interpreter will use.
\begin{lstlisting}
interpreter inputLine (parse keyword) {requiredArgument} [(first option) {firstOptionArgument} [(second option) {secondOptionArgument}]]
\end{lstlisting}
There is no case sensitivity in this line.  The initial part of this
line after the specifiers tells the generator what the required parts
of the line are and what the keywords are that trigger the parser.
This part is
\begin{lstlisting}
(parse keyword) {requiredArgument}
\end{lstlisting}
The phrase that's contained in parentheses is the keyword that
triggers the parser.  The string in curly braces defines the argument
to the required part of the line.  There can be any number of
arguments to any part of the line, but they must each be defined by a
single, unspaced string.  A simple example is the input of the state,
or initial guess, file that will be used in a Charon job.  The
generator input line could be
\begin{lstlisting}
interpreter inputLine (Import State File) {filename}
\end{lstlisting}
This tells the parser to look for a the keywords ``import state file''
and use ``filename'' as an argument to define parameter list
entries.  For example,
\begin{lstlisting}
Import state file myInitialGuess.exo
\end{lstlisting}
in a Charon input file will create parameter list elements that tell
Charon to look for the file ``myInitialGuess.exo'' as an initial
guess.  Import state file is not case sensitive, but the argument
``myInitialGuess.exo'' must be.

The parser can also be instructed to look for options on the line.
This is done by enclosing optional pieces and arguments in square
braces and nested as shown in the generic example.  A more concise
example for state files is
\begin{lstlisting}
interpreter inputLine (Import State File) {filename} [(at Index) {index}]
\end{lstlisting}
This tells the parser that there may be (not required) additional
instruction to use a particular time or parameter plane in the
imported state file and is indicated by the optional parse keywords
``at index.''  By continuing with the earlier example, a line of input
to the interpreter could be
\begin{lstlisting}
Import state file myInitialGuess.exo at index 3
\end{lstlisting}
will tell Charon to read in a file called myInitialGuess.exo and to
use the third time plane in that file as the initial guess to the
simulation.  In this case, ``at index 3'' can be omitted and certain
default behaviors will be employed.  Those are defined in the second
group of the input file that defines the parameter list entries.

The second group of inputs all begin with an ``xml'' prefix regardless
of whether the interpreter is ultimately instructed to generate XML
formatted parameter lists or ``yaml'' formatted parameter lists.  The
one that must {\bf always} be included are the lines that are
indicated by ``xmlRequired.''  An example of a required line is
\begin{lstlisting}
xmlRequired Charon->Mesh->Exodus File,File Name,string,{filename}
\end{lstlisting}
The xmlRequired lines will map directly from the non square bracketed
parts of the interpreter input line to xml formatted parameter list.
In general, a single line interpreter input may map to multiple lines
of parameter lists.  Each line of the parameter list that maps to
required input must be separately defined and so there is no limit to
the number of xmlRequired lines that can be specified.  The only
necessary thing is that arguments to the required interpreter input
line must appear in at least one of the xmlRequired lines.  In this
case, it is the ``filename'' argument.  In the xmlRequired line---and
all other xml lines described later---the initial part of the line is
case insensitive.  Everything following that defines the parameter and
how it is nested in the parameter list is case sensitive because
parameter lists are case sensitive.

The format of the parameter list line follows from the requirements of
Teuchos \cite{teuchosCitation} parameter lists and parameters.  Each
parameter must have a name, type and value.  These are defined in the
interpreter by the comma separated section of the example above.  In
this case, ``File Name'' is the parameter name, ``string'' is the
parameter type and ``{filename}'' is the parameter value which will be
filled in by the argument value supplied by the user in the
interpreter input.  In other words, the xmlRequired line above will
produce the following XML formatted parameter list output,
\begin{lstlisting}
<ParameterList name="Charon">
  <ParameterList name="Mesh">
    <ParameterList name="Exodus File">
      <Parameter name="File Name" type="string" value="{filename}" />
    </ParameterList>
  </ParameterList>
</ParameterList>
\end{lstlisting}

The nesting of the xml output is defined by the parts of the required
line that are separated by the arrow, ``->'' characters as seen in the
filename example.

Optional inputs are very similar but but the possibility of multiple
options requires additional information---namely a parse keyword
associated with the option.  An example of an optional line that is
consistent with earlier examples is,
\begin{lstlisting}
xmlOptional (at Index) Charon->Mesh->Exodus File,Restart Index,int,{index}
\end{lstlisting}
The syntax in this case is identical to the required syntax except
that the parse keyword ``at index'' has been included.  Other options
will have separate keywords.  As in the required line, there may be
any number of xmlOptional lines connected to a single option in the
interpreter input.  This xmlOptional line will create the XML
formatted lines,
\begin{lstlisting}
<ParameterList name="Charon">
  <ParameterList name="Mesh">
    <ParameterList name="Exodus File">
      <Parameter name="Restart Index" type="int" value="{index}" />
    </ParameterList>
  </ParameterList>
</ParameterList>
\end{lstlisting}
Note that this optional input will {\bf only} be included in the
parameter list if the user supplies the option.  Otherwise it will not
appear.

The final category in the xml input group are xmlDefault parameters.
These parameters are always included in the ultimate parameter list
and never contain any arguments.  For example, Charon parameter list
elements that contain the state guess file must also define the initial
file format.  For Charon, this never changes from the exodus file
format and as such requiring it in input is a waste of the user's time
and serves only to make for long, confusing input prone to error.
xmlDefault parameters are the cure for this.  Consistent with the
earlier examples,
\begin{lstlisting}
xmldefault  Charon->Mesh,Source,String,Exodus File
\end{lstlisting}
and tells Charon that the state file type will always be an exodus
file.  This line produces the XML formatted parameter list,
\begin{lstlisting}
<ParameterList name="Charon">
  <ParameterList name="Mesh">
    <Parameter name="Source" type="string" value="Exodus File" />
    </ParameterList>
  </ParameterList>
</ParameterList>
\end{lstlisting}

To put this all together, the createLineParser input file,
\begin{lstlisting}
interpreter name ImportStateFile

interpreter inputLine (Import State File) {filename} [(at Index) {index}]

xmlRequired Charon->Mesh->Exodus File,File Name,string,{filename}

xmlOptional (at Index) Charon->Mesh->Exodus File,Restart Index,Int,{index}

xmlOptional (second Option) Charon->Mesh->Exodus File,Option Two,String,{option2}

xmldefault  Charon->Mesh,Source,String,Exodus File
xmldefault  Charon->Mesh->Exodus File,Restart Index,Int,-1
\end{lstlisting}
will create a parser that will take the interpreter input line
\begin{lstlisting}
import state file powerMOSFET.sg.Vdrain.exo at index 7
\end{lstlisting}
that will produce the XML formatted parameter list
\begin{lstlisting}
<ParameterList name="Charon">
  <ParameterList name="Mesh">
    <Parameter name="Source" type="string" value="Exodus File" />
    <ParameterList name="Exodus File">
      <Parameter name="File Name" type="string" value="powerMOSFET.sg.Vdrain.exo" />
      <Parameter name="Restart Index" type="int" value="7" />
    </ParameterList>
  </ParameterList>
</ParameterList>
\end{lstlisting}

One final comment must be made about default lines {\em vis-a-vis}
optional lines.  In this example, the ``Restart Index'' parameter has
been included in both default and optional input.  Parameter list
entries are prioritized such that default entries will always be
included, but will always be replaced by optional entries when they
are specified.  In Charon, the restart index need not always be
supplied.  When it isn't, a restart index of 0 is implied.  That is,
the xmlDefault ``Restart Index'' could have been left out without
penalty.  It is ultimately up to the developer how explicit they feel
the parameter list should be.  The author prefers more explicit to less
as the user won't normally be burdened by the longer parameter list
input, but its presence could be found useful at times.

Lastly, the interpreter is set up to provide help to the user when
requested.  For this reason, help lines are included as well that are
not related to parameter lists, but only to interpreter input.  The
developer must always include the two entries,
\begin{lstlisting}
interpreter shortHelp {Specify state file name and state index}
\end{lstlisting}
and
\begin{lstlisting}
interpreter longHelp {Specify the name of the file (exodus) that contains the states for input/initial guess. <> Optionally, specify the index of the time plane or parameter plane if the file contains multiple states. <> filename is the name of the file.  It is case sensitive. <> index is the integer index of the state to be used as input.}
\end{lstlisting}
edited appropriately for the input line.  With the interpreter help
option, the line itself will be echoed absent the () characters along
with the requested verbosity. The short help line should always be
succinct and in active voice.  The long help line is less rigidly
defined, but in general ought to be a more comprehensive description
of the input and should always contain a description of the arguments
in the line.  The ``<>'' characters inserted into the long help string
connote a line break for more attractive, formatted output of the help
line.

\subsubsection{Optional Priority Specification}

All of the xml lines in the parser input file will have a default
priority associated with it.  xmlRequired and xmlOptional each default
to a priority of 2.  xmlDefault lines default to a priority of 1.  The
higher the number, the higher the priority.  This is illustrated in
the import state file parser with the restart index.  The default
value for the index is -1.  If the user specifies the index in the
input file to be 2, for example, the xmlOptional line that handles
that option takes priority over the xmlDefault line that sets the
index to -1 and so ultimately, the index is set equal to 2.

There are occasions when more than one parser generates the same line
of xml parameter input that may conflict.  An example of this is the
ohmic BC line,
\begin{lstlisting}
BC is ohmic for drain on bulk swept from 0 to 2
\end{lstlisting}
This line creates a sweep of the voltage on the drain contact to vary
from 0 to 2.  The default initial step size in the sweep will be 1
volt.  This might be too large a step size to start with.  For this
reason, there is a sweep options block wherein the user can modify the
behavior of the sweep.  For example,
\begin{lstlisting}
start sweep options
      initial step size = 0.5
end sweep options
\end{lstlisting}
This modifies the size of the initial step in the sweep.  A problem
arises when the sweep options block appears in the input file before
the BC line.  The initial step size in the sweep options block will
get overridden by the step size in the BC parser.  The reason for this
is that the default initial step size specified in the BC parser is
actually a part of the xmlOptional behavior for a sweep of a BC.  So
the two parameter inputs will have the same priority.  The best way to
handle these situations is to use an optional custom priority for the
initial step size parameter in the sweep options block.  It is
specified thus,
\begin{lstlisting}
xmlRequired Charon->Solution Control->LOCA->Step Size,Initial Step Size,double,{stepSize} priority 5
\end{lstlisting}
If the priority keyword does not exist in the previous line, the
default priority of 2 will be assigned that parameter.  By adding the
\texttt{priority 5} addendum to the line, one guarantees that a higher
priority is assigned to this specification and will override the BC
step size specification regardless of the order of appearance in the
input file.

\subsection{Charon Create Block Parser}

In many instances, it is most convenient to organize input into
blocks.  The Charon interpreter allows for this, but it is encouraged
that nesting of these blocks be kept to a minimum depth.  The
generateInterpreter script is instructed to create parsers of block
input in a very similar way that line parsers are created.  The
difference is organizational.  Block input files are read from the
same parseInputs subdirectory as line parser inputs, but all contain
the .blockinp suffix.  An example of a material model block parser
generator input is,
\begin{lstlisting}
interpreterBlock name MaterialBlock

interpreterBlock (start Material Block)
\end{lstlisting}
This tells generateInterpreter to create a blockParser class called
charonBlockParserMaterialBlock in a file called
charonBlockParserMaterialBlock.py in the parsers directory.  The
keywords in the interpreter file that open the block are ``start
Material Block'' (not case sensitive).  It is implied that ``end
Material Block'' will close the block in the interpreter file and must
be provided by the user.  Each block must have parameter lines inside.
A separate input file that defines the parser for each of those lines
as described in the previous subsection must be included.  The line
parser input are organized into subdirectories that mimic the nesting
of the interpreter block parser.  If the MaterialBlock parser in the
example is defined in the .blockinp file, it is implied that there is
also a subdirectory called parseInputs/MaterialBlock.  This
subdirectory includes all of the line parsers that are defined in the
block.  For example, if the material is to be named, the
parseinputs/MaterialBlock directory will contain a line parser input
file called createMaterialName.inp.  The creation of the MaterialBlock
block parser in the parsers directory will also create a
parsers/MaterialBlock subdirectory that contains all line parsers
associated with that first, nested level of the block.  In other
words, the input file parseInputs/MaterialBlock/createMaterialName.inp
will create parsers/MaterialBlock/charonLineParserMaterialName.py.

It is possible for blocks of interpreter input to contain other blocks
of input.  That is, the parseInputs/MaterialBlock directory might
contain a createDoping.blockinp file that provides for a block of line
parsers in that block.  Naturally then, there will be a directory
called parseInputs/MaterialBlock/Doping that has in it line parser
inputs that defined the parsers for each line in the doping block.  To
lay it out, there is a block parser input file called
createMaterialBlock.blockinp,
\begin{lstlisting}
interpreterBlock name MaterialBlock

interpreterBlock (start Material Block)
\end{lstlisting}
which creates a block parser invoked by the ``start Material Block''
keywords.  There are also files in the parseInputs/MaterialBlock
directory called createMaterialName.inp,
\begin{lstlisting}
interpreter name MaterialName

interpreter inputLine (material name) = {materialName}

interpreter shortHelp { }

interpreter longHelp { }
\end{lstlisting}
as well as another block parser input called createDoping.blockinp,
\begin{lstlisting}
interpreterBlock name Doping

interpreterBlock (start Doping) 
\end{lstlisting}
This implies another subdirectory parseInputs/MaterialBlock/Doping
that contains line parsers for the doping.

An interpreter input line might read something like,
\begin{lstlisting}
start Material Block
    material name = Si
    start Doping
      ....
    end Doping
end Material Block
\end{lstlisting}
All parameter list input will be generated according to the individual
line parsers as described in the previous section.  The generated
parsers are organized into the parsers directory with the same
subdirectory structure as the parseInputs directory structure.



\subsection{Charon Create Modifiers}



The parsers created by the generate interpreter script and all
associated inputs are relatively simple mappings.  They translate
simple, easy to understand user input to the multiple lines of nested
xml formatted input that is native to Charon/Teuchos parameter lists.
They contain very little logic.

It is sometimes necessary to include logic in a parser.  This is
possible in the Charon interpreter by using special modifier scripts,
though the inputs should be designed where there is as little reliance
on modifiers as possible.  One example is doing parameter sweeps.  The
input line for a contact boundary condition may read as follows:
\begin{lstlisting}
BC is contact on insulator for gate on gateoxide with work function 4.0 swept from 0 to 2
\end{lstlisting}
Clearly, the user is requesting a sweep from 0 to 2 volts on the gate
contact of a FET.  In the xml input for Loca, the range is specified
as a minimum and maximum value and there is a signed initial step size
that determines the direction of the sweep---from 0 to 2 in a positive
direction in this case.

If the requested boundary condition and associated voltage sweep is
slightly different, say,
\begin{lstlisting}
BC is contact on insulator for gate on gateoxide with work function 4.0 swept from 0 to -2
\end{lstlisting}
the simple mapping breaks down.  The minimum Loca parameter value will
be set to 0 and the maximum to -2 with a sweep in the wrong direction.
These special cases where a simple mapping is not possible, the
interpreter allows for developer-defined modifiers to produce the
correct behavior.  This is the one instance where the developer must
write Python code when developing parsers for the interpreter.

All of the Python code the developer must write is contained in a
function titled {\em testForModification}.  In principle, a parser may
have an unlimited number of modifiers, but they should be kept to a
minimum to avoid convoluting the interpreter.  Any modifier will
always be associated with a single, specific parser and a single file
contains all modifiers that will be associated with that parser.  The
file must have the same canonical name as its associated parser. E.g.,
the parser input file for the boundary condition described here is
{\texttt{createContactOnInsulator.inp}}.  The associated files that
define the modifiers must then be named
\texttt{createContactOnInsulator.modifierinp.py}. All modifiers
associated with this parser will be defined in the file of this name.
Each modifier will be a function called \texttt{testForModification}.
Each identically named function is contained inside a block that
starts \texttt{start modifier X} and ends \texttt{end modifier X}
where \texttt{X} is a unique integer number for the modifier.  For
example, the Python code that handles the voltage sweep input
described herein is:
\begin{lstlisting}
start Modifier 0

def testForModification(self,pLList):

    foundMinValue = False
    foundMaxValue = False
    foundStepSize = False
    makeMinMaxModification = False
    makeStepSizeModification = False
        
    for lineNumber, line in enumerate(pLList):

        # Capture the min value
        if line.find("Charon->Solution Control->LOCA->Stepper,Min Value,double") >= 0:
            lineParts = line.split(",")
            minValue = lineParts[-1]
            minValueLine = lineNumber
            foundMinValue = True

        # Capture the max value
        if line.find("Charon->Solution Control->LOCA->Stepper,Max Value,double") >= 0:
            lineParts = line.split(",")
            maxValue = lineParts[-1]
            maxValueLine = lineNumber
            foundMaxValue = True

        # Capture the initial step size
        if line.find("Charon->Solution Control->LOCA->Step Size,Initial Step Size,double") >= 0:
            lineParts = line.split(",")
            initialStepSize = lineParts[-1]
            initialStepSizeLine = lineNumber
            foundStepSize = True

    if foundMinValue == True and foundMaxValue == True:
        if float(minValue) > float(maxValue):
            makeMinMaxModification = True
            replacementMinLine = "Charon->Solution Control->LOCA->Stepper,Min Value,double,"+maxValue
            replacementMaxLine = "Charon->Solution Control->LOCA->Stepper,Max Value,double,"+minValue

    if makeMinMaxModification and float(initialStepSize) > 0:
        makeStepSizeModification = True
        newStepSize = str(-float(initialStepSize))
        replacementStepSizeLine = "Charon->Solution Control->LOCA->Step Size,Initial Step Size,double,"+newStepSize


    # Make modifications
    if makeMinMaxModification == True:
        pLList[minValueLine] = replacementMinLine
        pLList[maxValueLine] = replacementMaxLine

    if makeStepSizeModification == True:
        pLList[initialStepSizeLine] = replacementStepSizeLine

    return pLList

end Modifier 0
\end{lstlisting}
Should this parser require a second modifier, it would be contained in
\texttt{start modifier 1} and \texttt{end modifier 1} and the function
name would still be named \texttt{testForModification}.  The
generateInterpreter script will create the necessary boiler plate
required to generate modifier script for the interpreter.

The modifiers will only be executed when their use is called out in
the parser itself.  To do this, a special line must be specified in
the parser input files.  The parser input file for the contact BC is
\begin{lstlisting}

interpreter name ContactOnInsulator

interpreter inputLine (BC is contact on insulator for) {sidesetID} on {geometryBlock} with work function {workFunction} [(fixed at) {potential}[ (swept from) {potential1} to {potential2}]]

interpreter shortHelp {Specify the potential on a contact}

interpreter longHelp {Specify the potential on a contact. <> sidesetID is the contact name/type <> geometryBlock is the geometry name the contact is attached to <> potential is the value in volts}

xmlRequired Charon->Boundary Conditions->{sidesetID}ANONYMOUS,Type,string,Dirichlet
xmlRequired Charon->Boundary Conditions->{sidesetID}ANONYMOUS,Sideset ID,string,{sidesetID}
xmlRequired Charon->Boundary Conditions->{sidesetID}ANONYMOUS,Element Block ID,string,{geometryBlock}
xmlRequired Charon->Boundary Conditions->{sidesetID}ANONYMOUS,Equation Set Name,string,ELECTRIC_POTENTIAL
xmlRequired Charon->Boundary Conditions->{sidesetID}ANONYMOUS,Strategy,string,Contact On Insulator
xmlRequired Charon->Boundary Conditions->{sidesetID}ANONYMOUS->Data,Work Function,double,{workFunction}

xmlOptional (fixed at) Charon->Boundary Conditions->{sidesetID}ANONYMOUS->Data,Voltage,double,{potential}

# Set the data parameter to a string
xmlOptional (swept from) Charon->Boundary Conditions->{sidesetID}ANONYMOUS->Data,Varying Voltage,string,Parameter

# Modify the solver type from NOX to LOCA
xmlOptional (swept from) Charon->Solution Control,Piro Solver,string,LOCA

#LOCA Parameters
xmlOptional (swept from) Charon->Solution Control->LOCA->Predictor,Method,string,Constant
xmlOptional (swept from) Charon->Solution Control->LOCA->Stepper,Continuation Method,string,Natural
xmlOptional (swept from) Charon->Solution Control->LOCA->Stepper,Initial Value,double,{potential1}
xmlOptional (swept from) Charon->Solution Control->LOCA->Stepper,Continuation Parameter,string,Varying Voltage
xmlOptional (swept from) Charon->Solution Control->LOCA->Stepper,Max Steps,int,1000
xmlOptional (swept from) Charon->Solution Control->LOCA->Stepper,Max Value,double,{potential2}
xmlOptional (swept from) Charon->Solution Control->LOCA->Stepper,Min Value,double,{potential1}
xmlOptional (swept from) Charon->Solution Control->LOCA->Stepper,Compute Eigenvalues,bool,0
xmlOptional (swept from) Charon->Solution Control->LOCA->Step Size,Initial Step Size,double,1.0
xmlOptional (swept from) Charon->Solution Control->LOCA->Step Size,Aggressiveness,double,1.0

# Set the parameters block
xmlOptional (swept from) Charon->Active Parameters,Number of Parameter Vectors,int,1
xmlOptional (swept from) Charon->Active Parameters->Parameter Vector 0,Number,int,1
xmlOptional (swept from) Charon->Active Parameters->Parameter Vector 0,Parameter 0,string,Varying Voltage
xmlOptional (swept from) Charon->Active Parameters->Parameter Vector 0,Initial Value 0,double,{potential1}
xmlOptional (swept from) use Modifier 0
\end{lstlisting}

This input creates a parser that specifies the voltage on the contact
that is either fixed or swept from one voltage to another.  The use
case described here is for a voltage sweep that may be swept from high
to low or low to high.  The optional argument \texttt{swept from}
requires the modifier to guarantee proper execution and so the line
\texttt{use Modifier 0} is added to that option.  This will trigger
the interpreter to execute that modifier when the sweep option is
selected by the user.

No modifier is executed until the input file has been completely
parsed.  The parameter list that is sent to the modifier script for
modification will be the completed one for the current job.

 %Charon parsers



\section{Syntax Guidance}

To set a value for a specific parameter,
\begin{lstlisting}
Parameter = 1.0
\end{lstlisting}

To set a condition or model for a parameter,
\begin{lstlisting}
Parameter is Vandalay cubic
\end{lstlisting}

 %syntax



\section{Tutorials}


This section contains several tutorials on how to create parsers.  It
starts with a tutorial on a very simple line parser with one optional
section. Simple block parsers with named blocks and nested line
parsers are next.  Finally, advanced line parsers with custom priority
and logic modifiers are last.  The details of the input files to
create line and block parsers are described in the initial chapter of
this document.  The purpose of this chapter is to take a developer
step-by-step through the creation of a new parser using some of the
tools available for the purpose.  The developer should refer to
chapter 1 for deeper insight to the creation of a parser.

\subsection{Some Quick Notes on the Interpreter}

In the Charon tcad-charon repository, there are two main python
scripts associated with the interpreter: charonInterpreter.py in the
\texttt{scripts/charonInterpreter} directory and
generateInterpreter.py in the
\texttt{scripts/charonInterpreter/parseGenerator} directory.  The
former is the user end main script and the latter the developer end
main script.

The charonInterpreter.py script may be invoked with multiple options.
The one that is almost always used is the \texttt{--input} option.
This is simply the option to use a specific input file for an
interpreter run.  Executing
\begin{lstlisting}
charonInterpreter.py --input input.inp
\end{lstlisting}
will process the contents of the \texttt{input.inp} file into the xml
formatted parameter list file that is native to Charon and stops
there.  The filename will be \texttt{input.inp.xml}.  Charon may be
run through the mpirun command with that xml file if desired.

A Charon run may also be executed with the interpreter script with
additional options.  For example,
\begin{lstlisting}
charonInterpreter.py --input input.inp --np 4 --run
\end{lstlisting}
This will process \texttt{input.inp} into a parameter list and then
execute Charon on that parameter list on 4 processors.  If the Charon
executable is located in a place included in the user's \texttt{path}
variable, the script will find it and use it.  If not, the
\texttt{CHARON\_EXECUTABLE\_PATH} environment variable may be set that
contains the path to the executable, for example,
\begin{lstlisting}
export CHARON_EXECUTABLE_PATH=/home/lcmusso/TCAD/build/src
\end{lstlisting}
There is also an implicit assumption that the name of the executable
is \texttt{charon\_mp.exe}.  If it is not, the
\texttt{CHARON\_EXECUTABLE} environment variable may be set to force
the interpreter to look for an otherwise named charon executable.

The generateInterpreter.py script is usually invoked with no options.
It will automatically process all of the parser generator input files
located in
\texttt{scripts/charonInterpreter/parseGenerator/parseInputs} and all
of its subdirectories into line and block parsers and all of the
modifiers and parser libraries required by the charon interpreter.  Of
the of parsers will be created in the
\texttt{scripts/charonInterpreter/parsers} directory and will have a
subdirectory structure that mirrors the directory structure under
\texttt{parseInputs.}


\subsection{Creating a Simple Line Parser}

Whether a developer is creating a parser for a new or existing
feature, the parameter list almost always antedates the interpreter
parser.  That is, a developer will ``work the kinks out'' modifying
parameter list input prior even to thinking about the interpreter.  So
an xml formatted parameter list is generally the place to start when
creating a parser to handle this part of the parameter list.

An input file to generate a parser consists of two major subsections.
The first is every line that starts with the \texttt{interpreter}
keyword and the second is every line that starts with the
\texttt{xml*} family of keywords.  \texttt{interpreter} keywords can be
thought of as a front end of the interpreter.  They define the name of
the parser, the syntax the developer wants the user to employ and all
of the help information the developer wishes the user to see.  All of
the keywords that start with \texttt{xml} are ones which define the
parameter lists that are the target mapping of the user's input line.
Chapter 1 defines each of these in detail.

A full-blown Charon input file is not required to work on a new
parser.  One should always start with the parameter list entries
required, create the parser generator input file and iterate until the
parser reproduces the xml parameter list faithfully.  There are two
tools available to speed this process.

A good example of a simple parser is importing a state file.  One of
the first things commonly seen in a Charon parameter list is the named
``initial guess'' file.  In interpreter parlance, this is called a
state file.  The xml formatted parameter list is:
\begin{lstlisting}
<ParameterList name="Charon">
    <ParameterList name="Mesh">
        <Parameter name="Source" type="string" value="Exodus File" />
        <ParameterList name="Exodus File">
            <Parameter name="File Name" type="string" value="bjt2d_equ.exo" />
            <Parameter name="Restart Index" type="int" value="3" />
        </ParameterList>
    </ParameterList>
</ParameterList>
\end{lstlisting}
So what's known about this parser is that it must contain sensible
syntax to specify a file name and a restart index.  And it may be
desirable to make the restart index an optional part of the line.

The first thing to do is to create the input file for the parser.  In
this case, the file \texttt{createImportStateFile.inp} is created in
the \texttt{parseInputs} subdirectory.  The first line in that file
gives a unique name to the parser to be created:
\begin{lstlisting}
interpreter name ImportStateFile
\end{lstlisting}
In practice, it is preferred to have the name of the parser match the
name of the input file with a \texttt{create} prefix and a
\texttt{.inp} suffix.  Note that the parser generated will be located
in the \texttt{parsers} directory with the name
\texttt{charonLineParserImportStateFile.py} with the
\texttt{ImportStateFile} part of it coming from the name of the
parser, not the name of the parser input file name.  This keeps things
systematic and easy to find.

The second line of the parser input file will contain the syntax the
developer wishes the user to employ.  Let's say the line should be
\begin{lstlisting}
import state file bjt2d_equ.exo at index 3
\end{lstlisting}
Let's say furthermore that the only required part of this line be the
specification of the state file name and that the restart index be
optional.  Moreover, we need to select keyword sequences which will
uniquely identify this line to the parser for both the required and
optional parts of this line.  Keywords of the line are enclosed in
parentheses and optional parts of the line are enclosed in square
brackets and include a parenthetical part that contains the keywords
for the option.  Putting this together, the second line of the parser
input file becomes
\begin{lstlisting}
interpreter inputLine (Import State File) {filename} [(at Index) {index}]
\end{lstlisting}
where the keywords \texttt{interpreter inputline} tell the
generateInterpreter script to process this line as interpreter syntax.
The \texttt{(import state file)} section tells the interpreter the
keyword sequence to trigger the parser.  The \texttt{\{filename\}} is
a variable for the name of the file the user wishes to import.  It was
decided that the restart index should be optional and so the section
of the line that pertains to the specification of the restart index is
enclosed in square brackets and necessarily contains the parenthetical
\texttt{(at index)} which triggers the option processing in the
parser.  Note that on the interpreter end, only \texttt{filename} is
case sensitive.  Everything else may be any combination of upper and
lower case except for \texttt{index} which must be an integer.

Two additional lines with the \texttt{interpreter} keyword contain
help information for the user.  In this example,
\begin{lstlisting}
interpreter shortHelp {Specify state file name and state index}

interpreter longHelp {Specify the name of the file (exodus) that contains the states for input/initial guess. <> Optionally, specify the index of the time plane or parameter plane if the file contains multiple states. <> filename is the name of the file.  It is case sensitive. <> index is the integer index of the state to be used as input.}
\end{lstlisting}
provide a pithy description of what this interpreter line means and a
more detailed description that includes a description of optional
inputs.  These are invoked by the user issuing the \texttt{--help} or
\texttt{--longhelp} options to the charonInterpreter.py script.


Now in the second major subsection of the parser input file, the
parameter list and how it maps from the interpreter line must be
specified so that the xml formatted input seen above is created.
Internally, the interpreter does not store parameters in an xml
format.  It has its own format for parameter lists in which the
nesting information accompanies every parameter.  Generally, the
format is for example
\begin{lstlisting}
Nest0->Nest1->Nest2,parameter name, parameter type, parameter value
\end{lstlisting}
The xml parameter list seen above must be translated into the
interpreter format for the parser input file. For this simple example,
the manual translation from xml to interpreter is relatively easy.
However, when the interpreter line maps to dozens of parameters with a
complicated nesting, such as might be seen for solver parameters, this
process is tedious and error prone.  In the
\texttt{scripts/charonInterpreter/tools} directory, there is a script
called \texttt{xmlToLCM.py} which does this translation automatically.
This script will take a text file \texttt{xmlbjt.xml} which contains
the xml formatted parameter list seen above and create a new text file
called \texttt{xmlbjt.xml.LCMified} which contains the translated
interpreter formatted parameter list, viz
\begin{lstlisting}
Charon->Mesh,Source,string,Exodus File
Charon->Mesh->Exodus File,File Name,string,bjt2d_equ.exo
Charon->Mesh->Exodus File,Restart Index,int,3
\end{lstlisting}
These lines must be categorized as default, required and optional and
added to the parser input file appropriately.

The \texttt{Source} parameter in the first line,
\begin{lstlisting}
Charon->Mesh,Source,string,Exodus File
\end{lstlisting}
must always be present in an identical way.  It must not ever vary the
way Charon works now.  It is nothing but a useless burden to require a
user to put this into every input file they create.  This is an ideal
candidate for a default line.  The interpreter will always write it
into the parameter list and the user need never see it.  This is
specified in the parser input file as
\begin{lstlisting}
xmlDefault Charon->Mesh,Source,string,Exodus File
\end{lstlisting}
The name of the file the user wishes to import must always be
specified and is never a default name.  This is a good candidate for a
required input and is specified in the parser input file as
\begin{lstlisting}
xmlRequired Charon->Mesh->Exodus File,File Name,string,{filename}
\end{lstlisting}
where the filename variable replaces the explicit file name in the
example and ties this particular parameter to the interpreter input.
That is, \texttt{\{filename\}} gets replaced in the parameter list by
the string specified in the interpreter input file.

Lastly, we wish to include the restart index as an optional parameter.
Moreover, we may also wish for the restart index to take on a default
value if the option is not specified by the user.  The default value
for the restart index is specified in the same way the \texttt{Source}
parameter was specified,
\begin{lstlisting}
xmlDefault Charon->Mesh->Exodus File,Restart Index,int,-1
\end{lstlisting}
This creates a parameter \texttt{Restart Index} with a value of -1.
To create the optional parameter, the parser input syntax is similar
to default and required parameters, but must also contain the optional
keywords,
\begin{lstlisting}
xmlOptional (at index) Charon->Mesh->Exodus File,Restart Index,int,{index}
\end{lstlisting}
so the \texttt{Restart Index} appears twice in the parser input file.
In one it contains default value information and in the other,
optional user-supplied information.  Here again, the
\texttt{\{index\}} variable maps back to the interpreter input line.

By pulling all of this together, the entire parser input file now reads:
\begin{lstlisting}
interpreter name ImportStateFile

interpreter inputLine (Import State File) {filename} [(at Index) {index}]

interpreter shortHelp {Specify state file name and state index}

interpreter longHelp {Specify the name of the file (exodus) that contains the states for input/initial guess. <> Optionally, specify the index of the time plane or parameter plane if the file contains multiple states. <> filename is the name of the file.  It is case sensitive. <> index is the integer index of the state to be used as input.}

xmlRequired Charon->Mesh->Exodus File,File Name,string,{filename} 

xmlOptional (at Index) Charon->Mesh->Exodus File,Restart Index,int,{index}

xmldefault  Charon->Mesh,Source,string,Exodus File
xmldefault  Charon->Mesh->Exodus File,Restart Index,int,-1
\end{lstlisting}
The parser is then generated by executing the generateInterpreter.py
script.

The final step the developer must execute is to validate the parser by
running the charon interpreter on a file that contains the new
interpreter line.  Say our file is \texttt{bjt\_2d.inp} and has the
line
\begin{lstlisting}
import state file bjt2d_equ.exo at index 3
\end{lstlisting}
Executing
\begin{lstlisting}
charonInterpreter.py --input bjt_2d.inp
\end{lstlisting}
produces a file named \texttt{bjt\_2d.inp.xml} with the contents
\begin{lstlisting}
<ParameterList name="Charon" >
  <ParameterList name="Mesh" >
    <ParameterList name="Exodus File" >
      <Parameter name="Restart Index" type="int" value="3" />
      <Parameter name="File Name" type="string" value="bjt2d_equ.exo" />
    </ParameterList>
    <Parameter name="Source" type="string" value="Exodus File" />
  </ParameterList>
</ParameterList>
\end{lstlisting}
Other than ordering of lines, this parameter list is identical.  In
this simple example, this is easy to see. Once again, however, if this
parameter list were long and complicated, the validation will be
tedious and error prone.  Another script in the tools directory will
make the comparison and provide a kind of diff between the two files.
It reports what exists in one file that does not exist in the other.
In this case, the script \texttt{compareParameterLists.py} can be used
to compare \texttt{bjt\_2d.inp.xml} and the file that contained the
original xml input we worked from to get the following report:
\begin{lstlisting}
compareParameterLists.py bjt_2dOriginal.xml bjt_2d.inp.xml
The following is a list of items contained in bjt_2dOriginal.xml, but not in bjt_2d.inp.xml



The following is a list of items contained in bjt_2d.inp.xml, but not in bjt_2dOriginal.xml

\end{lstlisting}
If something went wrong with the mapping in the interpreter and the
xml generated turned out to be
\begin{lstlisting}
<ParameterList name="Charon" >
  <ParameterList name="Mesh" >
    <ParameterList name="Exodus File" >
      <Parameter name="Restart Index" type="int" value="300" />
      <Parameter name="File Name" type="string" value="bjt2d_equ.exo" />
    </ParameterList>
    <Parameter name="Source" type="string" value="Exodus File" />
  </ParameterList>
</ParameterList>
\end{lstlisting}
the report would be
\begin{lstlisting}
The following is a list of items contained in bjt_2dOriginal.xml, but not in bjt_2d.inp.xml

Charon->Mesh->Exodus File,Restart Index,int,3


The following is a list of items contained in bjt_2d.inp.xml, but not in bjt_2dOriginal.xml

Charon->Mesh->Exodus File,Restart Index,int,300
\end{lstlisting}
which highlights the difference.  A more complicated example would be
if the nesting were incorrect.  The compare script is too naive to
figure out how nesting should have been done.  When it finds nested
parameters in one file and not in another, it stops ever reporting the
one contains a parameter list not contained in the other.  For
example, suppose instead of the parameter list \texttt{Exodus File},
the parser input file contained a typo, \texttt{Eoxdus File}.  The
compare script is not smart enough to figure out the typo, but will
indicate that the two files differ with the following report:
\begin{lstlisting}
The following is a list of items contained in bjt_2dOriginal.xml, but not in bjt_2d.inp.xml

Charon->Mesh->Exodus File


The following is a list of items contained in bjt_2d.inp.xml, but not in bjt_2dOriginal.xml

Charon->Mesh->Eoxdus File
\end{lstlisting}
Nothing below the level of \texttt{Exodus File} is reported, but
sufficient information is provided to the developer to indicate the
error.

\subsection{Creating a Block Parser}

The line parsers in the Charon interpreter are where the bulk of the
work is done in parsing and running a Charon job.  As was described in
the previous section, a single line of interpreter input can map to
many lines of parameter lists.  It still often makes sense to bundle
line parsers into sensible groups.  This makes the input file neater,
more organized and easier to navigate.  These groups in the
interpreter are called blocks and they have their own parsers.  Block
parsers are in principle simpler than line parsers.  Their sole
purpose for existence is for organization and to make data input
simpler.  A good example of a block in the Charon input is where
material models are spelled out for a common region of a device, e.g.
\begin{lstlisting}
start Material Block siliconParameter
      material name is Silicon
      relative permittivity = 11.9
      start step junction doping
      	    acceptor concentration = 1e16
	    donor concentration = 1e16
	    junction location = 0.5
	    dopant order is PN
      end step junction doping
end Material Block siliconParameter
\end{lstlisting}
In this instance, two blocks are nested.  There is no limitation to
the depth that blocks may be nested, but it should be limited to as
few layers as makes sense to keep things simple.  Block parsers
control this nesting in an organized way and provide some of the
mapping information.  

Every line inside the blocks that begins neither with ``start'' nor
``end'' is just a line parser like the one created in the previous
section's tutorial.  The way the parser input files and thus the
parsers are organized mirrors the nested structure of the blocks.  So
the root directory for the parser input files,
\texttt{charonInterpreter/parseGenerator/parseInputs} contains the
block parser for the material block.  The structure and composition of
the block parser input file is similar though simpler than the line
parser input files.  The name of the block parser input file is
required only to have the suffix \texttt{blockinp}, but the following
convention has been used exclusively,
\begin{lstlisting}
create<BlockParserName>.blockinp
\end{lstlisting}
In this example, \texttt{BlockParserName} = \texttt{MaterialBlock} and
the first line in the block parser makes that specification,
\begin{lstlisting}
interpreterBlock name MaterialBlock
\end{lstlisting}
And just as in the line parser input file, there must be a line which
specifies the keywords that trigger the block parser,
\begin{lstlisting}
interpreterBlock (start Material Block) {MaterialBlockName}
\end{lstlisting}
where as before the key words are enclosed in parentheses and the
variable MaterialBlockName is enclosed in curly braces.  This
represents the minimum input to create a block parser.  It is often
the only input.

With blocks, there is a directory structure that accompanies and
mirrors the block structure.  Under the \texttt{parseInputs}
directory, the developer must create a subdirectory that has the same
name as the block parser.  In this example, it is the developer's
burden to create the\texttt{parseInputs/MaterialBlock} subdirectory.
Every line parser that accompanies every line in the block has an
input file that lives in this subdirectory.  That is, the line parsers
for
\begin{lstlisting}
  material name is Silicon
\end{lstlisting}
and 
\begin{lstlisting}
  relative permittivity = 11.9
\end{lstlisting}
namely \texttt{createMaterialName.inp} and
\texttt{createRelativePermattivity.inp} are located in the
\texttt{parseInputs/MaterialBlock} subdirectory.

This example contains one nested block layer.  The
directory-subdirectory structure simply repeats itself with nested
blocks.  There is a \texttt{createStepDoping.blockinp} block parser
input file and a \texttt{StepDoping} subdirectory in the
\texttt{MaterialBlock} block directory.  All line parse input files
related to the step doping block are located in the
\texttt{StepDoping} directory.

It is important to note that very commonly, parameters created by line
parsers in blocks almost always need to know the block name in order
to get the parameter list correct.  Every line parser inside the
blocks will have defined any variable specified in the block parser.
In this example, the variable \texttt{MaterialBlockName} is defined
and can be used in every line parser input file that exists under the
\texttt{MaterialBlock} directory tree.  The \texttt{MaterialName} line
parser input file is
\begin{lstlisting}
interpreter name MaterialName

interpreter inputLine (material name) is {materialName}

interpreter shortHelp {Set the Material name }

interpreter longHelp {Set the Material name <> materialName is the material in this material block (e.g. Silicon)}

xmlRequired Charon->Closure Models->{MaterialBlockName},Material Name,string,{materialName}
\end{lstlisting}

\subsection{Custom Parameter Priority}

It can happen on occasion that a parameter gets defined more than
once.  Most of the time, this presents no problem.  But sometimes it
does.  An example of this is doing potential sweeps on contacts.  An
interpreter input line for sweeping is
\begin{lstlisting}
BC is ohmic for anode on silicon swept from 0.5 to 0.25
\end{lstlisting}
The parser that processes this line will create the appropriate
parameter list for the boundary condition.  It also creates as
xmlOptional parameters the necessary inputs to switch the run from NOX
solver to LOCA solver and sets LOCA stepper information.  This parser
will set the initial step size of the sweep to 1.0.  A step size of 1
volt is often far too large to take and so a smaller step size should
be specified.  This can be done in an optional block for sweep
parameters,
\begin{lstlisting}
start sweep options
      initial step size = 0.25
end sweep options
\end{lstlisting}
With this, a parameter has been set that creates an initial step size
of 1.0 and a second one has been set that creates an initial step size
of 0.25.  This presents a potential problem because in the first
instance, the parameter is xmlOptional and the second, the parameter
is xmlRequired.  Optionals and Requireds have the same priority and
the one which will get used is the one last encountered in the list.
This makes the input file order dependent and undesirable.  This is
remedied by adding a custom priority tag to the desired specification.
When the user adds the optional sweep options block, anything
specified in there must take priority over anything else that is
default behavior.  The tag is set in the line parser input file for
the initial step size in the sweep option block as follows:
\begin{lstlisting}
interpreter name InitialStepSize

interpreter inputLine (Initial Step Size) = {stepSize}

interpreter shortHelp {Specify initial step size for parameter sweep}

interpreter longHelp {Specify initial step size for parameter sweep <> {stepSize} = the step size}

xmlRequired Charon->Solution Control->LOCA->Step Size,Initial Step Size,double,{stepSize} priority 5
\end{lstlisting}
If the priority tag is missing from the xmlRequired line, default
priority is assigned.  If the tag is present, the requested priority
is assigned.


\subsection{Anonymous Parameter Lists}

There are unfortunate instances in the Charon parameter lists that a
parameter list is unnamed.  This is called an anonymous parameter list
and it does create an organizational hazard.  The interpreter relies
on a unique nesting path to do its mapping from interpreter
specifications into parameter lists.  These are managed in a special
way in the interpreter and some care must be taken to get them right.
The parameter list name \texttt{ANONYMOUS} is always used where there
is an unnamed parameter list.  If the unnamed parameter list has no
peers, that is if it sits alone at its particular nesting level and
path, \texttt{ANONYMOUS} is all that is required to handle the mapping
correctly.  However, if there is more than one, \texttt{ANONYMOUS} is
not unique and the mapping will not be done correctly.  An important
example of this is contact boundary conditions.  The boundary
conditions on a diode are as follows:
\begin{lstlisting}
<ParameterList name="Charon">
    <ParameterList name="Boundary Conditions">
        <ParameterList>
            <Parameter name="Type" type="string" value="Dirichlet"/> 
            <Parameter name="Sideset ID" type="string" value="anode"/> 
            <Parameter name="Element Block ID" type="string" value="silicon"/> 
            <Parameter name="Equation Set Name" type="string" value="ALL_DOFS"/> 
            <Parameter name="Strategy" type="string" value="Ohmic Contact"/> 
            <ParameterList name="Data">
                <Parameter name="Voltage" type="double" value="0.5"/>
            </ParameterList>
        </ParameterList>    
        <ParameterList>
            <Parameter name="Type" type="string" value="Dirichlet"/> 
            <Parameter name="Sideset ID" type="string" value="cathode"/> 
            <Parameter name="Element Block ID" type="string" value="silicon"/> 
            <Parameter name="Equation Set Name" type="string" value="ALL_DOFS"/> 
            <Parameter name="Strategy" type="string" value="Ohmic Contact"/>
            <ParameterList name="Data">
                <Parameter name="Voltage" type="double" value="0"/>
            </ParameterList>
        </ParameterList>
    </ParameterList>
</ParameterList>
\end{lstlisting}
The anode and cathode are peer parameter lists and are unnamed and are
not unique.  Internally, Trilinos-Teuchos will replace anonymous
parameter lists with assigned names \texttt{child0}, \texttt{child1}
etc., but the user never sees it.  The interpreter can only handle
parameters with a unique path.  The way the interpreter manages this
is that any parameter list named \texttt{ANONYMOUS} will get written
to the mapped parameter list as unnamed.  This is true even if
\texttt{ANONYMOUS} represents only a fraction of the name.  The parser
input file in part for boundary conditions is
\begin{lstlisting}
interpreter name OhmicBC

interpreter inputLine (BC is ohmic for) {sidesetID} on {geometryBlock} [(fixed at) {potential}[ (swept from) {potential1} to {potential2}]]

interpreter shortHelp {Specify the potential on a contact}

interpreter longHelp {Specify the potential on a contact. <> sidesetID is the contact name/type <> geometryBlock is the geometry name the contact is attached to <> potential is the value in volts}

xmlRequired Charon->Boundary Conditions->{sidesetID}ANONYMOUS,Type,string,Dirichlet
xmlRequired Charon->Boundary Conditions->{sidesetID}ANONYMOUS,Sideset ID,string,{sidesetID}
xmlRequired Charon->Boundary Conditions->{sidesetID}ANONYMOUS,Element Block ID,string,{geometryBlock}
xmlRequired Charon->Boundary Conditions->{sidesetID}ANONYMOUS,Equation Set Name,string,ALL_DOFS
xmlRequired Charon->Boundary Conditions->{sidesetID}ANONYMOUS,Strategy,string,Ohmic Contact
\end{lstlisting}
Because the sidesetID that must be specified for the boundary
condition will be unique for that parameter list, it is prepended to
the ANONYMOUS part of the parameter.  This has the effect of giving
the interpreter a unique name to work with.  When the parameter list
is ultimately mapped and written, the entire
\texttt{{sidesetID}ANONYMOUS} string gets replaced with nothing and
becomes and unnamed parameter list.


 %Tutorials





    % ---------------------------------------------------------------------- %
    %
    %\appendix
    %\section{Historical Perspective}
    %\input{Charon_clusteringAppendix}


    \nocite{*}


    % ---------------------------------------------------------------------- %
    % References
    %
    \clearpage
    % If hyperref is included, then \phantomsection is already defined.
    % If not, we need to define it.
    \providecommand*{\phantomsection}{}
    \phantomsection
    \addcontentsline{toc}{section}{References}
    \bibliographystyle{plain}
    \bibliography{Charon_bibliography}


%    \printindex

    \include{SANDdistribution}

\end{document}
