

\section{Tutorials}


This section contains several tutorials on how to create parsers.  It
starts with a tutorial on a very simple line parser with one optional
section. Simple block parsers with named blocks and nested line
parsers are next.  Finally, advanced line parsers with custom priority
and logic modifiers are last.  The details of the input files to
create line and block parsers are described in the initial chapter of
this document.  The purpose of this chapter is to take a developer
step-by-step through the creation of a new parser using some of the
tools available for the purpose.  The developer should refer to
chapter 1 for deeper insight to the creation of a parser.

\subsection{Some Quick Notes on the Interpreter}

In the Charon tcad-charon repository, there are two main python
scripts associated with the interpreter: charonInterpreter.py in the
\texttt{scripts/charonInterpreter} directory and
generateInterpreter.py in the
\texttt{scripts/charonInterpreter/parseGenerator} directory.  The
former is the user end main script and the latter the developer end
main script.

The charonInterpreter.py script may be invoked with multiple options.
The one that is almost always used is the \texttt{--input} option.
This is simply the option to use a specific input file for an
interpreter run.  Executing
\begin{lstlisting}
charonInterpreter.py --input input.inp
\end{lstlisting}
will process the contents of the \texttt{input.inp} file into the xml
formatted parameter list file that is native to Charon and stops
there.  The filename will be \texttt{input.inp.xml}.  Charon may be
run through the mpirun command with that xml file if desired.

A Charon run may also be executed with the interpreter script with
additional options.  For example,
\begin{lstlisting}
charonInterpreter.py --input input.inp --np 4 --run
\end{lstlisting}
This will process \texttt{input.inp} into a parameter list and then
execute Charon on that parameter list on 4 processors.  If the Charon
executable is located in a place included in the user's \texttt{path}
variable, the script will find it and use it.  If not, the
\texttt{CHARON\_EXECUTABLE\_PATH} environment variable may be set that
contains the path to the executable, for example,
\begin{lstlisting}
export CHARON_EXECUTABLE_PATH=/home/lcmusso/TCAD/build/src
\end{lstlisting}
There is also an implicit assumption that the name of the executable
is \texttt{charon\_mp.exe}.  If it is not, the
\texttt{CHARON\_EXECUTABLE} environment variable may be set to force
the interpreter to look for an otherwise named charon executable.

The generateInterpreter.py script is usually invoked with no options.
It will automatically process all of the parser generator input files
located in
\texttt{scripts/charonInterpreter/parseGenerator/parseInputs} and all
of its subdirectories into line and block parsers and all of the
modifiers and parser libraries required by the charon interpreter.  Of
the of parsers will be created in the
\texttt{scripts/charonInterpreter/parsers} directory and will have a
subdirectory structure that mirrors the directory structure under
\texttt{parseInputs.}


\subsection{Creating a Simple Line Parser}

Whether a developer is creating a parser for a new or existing
feature, the parameter list almost always antedates the interpreter
parser.  That is, a developer will ``work the kinks out'' modifying
parameter list input prior even to thinking about the interpreter.  So
an xml formatted parameter list is generally the place to start when
creating a parser to handle this part of the parameter list.

An input file to generate a parser consists of two major subsections.
The first is every line that starts with the \texttt{interpreter}
keyword and the second is every line that starts with the
\texttt{xml*} family of keywords.  \texttt{interpreter} keywords can be
thought of as a front end of the interpreter.  They define the name of
the parser, the syntax the developer wants the user to employ and all
of the help information the developer wishes the user to see.  All of
the keywords that start with \texttt{xml} are ones which define the
parameter lists that are the target mapping of the user's input line.
Chapter 1 defines each of these in detail.

A full-blown Charon input file is not required to work on a new
parser.  One should always start with the parameter list entries
required, create the parser generator input file and iterate until the
parser reproduces the xml parameter list faithfully.  There are two
tools available to speed this process.

A good example of a simple parser is importing a state file.  One of
the first things commonly seen in a Charon parameter list is the named
``initial guess'' file.  In interpreter parlance, this is called a
state file.  The xml formatted parameter list is:
\begin{lstlisting}
<ParameterList name="Charon">
    <ParameterList name="Mesh">
        <Parameter name="Source" type="string" value="Exodus File" />
        <ParameterList name="Exodus File">
            <Parameter name="File Name" type="string" value="bjt2d_equ.exo" />
            <Parameter name="Restart Index" type="int" value="3" />
        </ParameterList>
    </ParameterList>
</ParameterList>
\end{lstlisting}
So what's known about this parser is that it must contain sensible
syntax to specify a file name and a restart index.  And it may be
desirable to make the restart index an optional part of the line.

The first thing to do is to create the input file for the parser.  In
this case, the file \texttt{createImportStateFile.inp} is created in
the \texttt{parseInputs} subdirectory.  The first line in that file
gives a unique name to the parser to be created:
\begin{lstlisting}
interpreter name ImportStateFile
\end{lstlisting}
In practice, it is preferred to have the name of the parser match the
name of the input file with a \texttt{create} prefix and a
\texttt{.inp} suffix.  Note that the parser generated will be located
in the \texttt{parsers} directory with the name
\texttt{charonLineParserImportStateFile.py} with the
\texttt{ImportStateFile} part of it coming from the name of the
parser, not the name of the parser input file name.  This keeps things
systematic and easy to find.

The second line of the parser input file will contain the syntax the
developer wishes the user to employ.  Let's say the line should be
\begin{lstlisting}
import state file bjt2d_equ.exo at index 3
\end{lstlisting}
Let's say furthermore that the only required part of this line be the
specification of the state file name and that the restart index be
optional.  Moreover, we need to select keyword sequences which will
uniquely identify this line to the parser for both the required and
optional parts of this line.  Keywords of the line are enclosed in
parentheses and optional parts of the line are enclosed in square
brackets and include a parenthetical part that contains the keywords
for the option.  Putting this together, the second line of the parser
input file becomes
\begin{lstlisting}
interpreter inputLine (Import State File) {filename} [(at Index) {index}]
\end{lstlisting}
where the keywords \texttt{interpreter inputline} tell the
generateInterpreter script to process this line as interpreter syntax.
The \texttt{(import state file)} section tells the interpreter the
keyword sequence to trigger the parser.  The \texttt{\{filename\}} is
a variable for the name of the file the user wishes to import.  It was
decided that the restart index should be optional and so the section
of the line that pertains to the specification of the restart index is
enclosed in square brackets and necessarily contains the parenthetical
\texttt{(at index)} which triggers the option processing in the
parser.  Note that on the interpreter end, only \texttt{filename} is
case sensitive.  Everything else may be any combination of upper and
lower case except for \texttt{index} which must be an integer.

Two additional lines with the \texttt{interpreter} keyword contain
help information for the user.  In this example,
\begin{lstlisting}
interpreter shortHelp {Specify state file name and state index}

interpreter longHelp {Specify the name of the file (exodus) that contains the states for input/initial guess. <> Optionally, specify the index of the time plane or parameter plane if the file contains multiple states. <> filename is the name of the file.  It is case sensitive. <> index is the integer index of the state to be used as input.}
\end{lstlisting}
provide a pithy description of what this interpreter line means and a
more detailed description that includes a description of optional
inputs.  These are invoked by the user issuing the \texttt{--help} or
\texttt{--longhelp} options to the charonInterpreter.py script.


Now in the second major subsection of the parser input file, the
parameter list and how it maps from the interpreter line must be
specified so that the xml formatted input seen above is created.
Internally, the interpreter does not store parameters in an xml
format.  It has its own format for parameter lists in which the
nesting information accompanies every parameter.  Generally, the
format is for example
\begin{lstlisting}
Nest0->Nest1->Nest2,parameter name, parameter type, parameter value
\end{lstlisting}
The xml parameter list seen above must be translated into the
interpreter format for the parser input file. For this simple example,
the manual translation from xml to interpreter is relatively easy.
However, when the interpreter line maps to dozens of parameters with a
complicated nesting, such as might be seen for solver parameters, this
process is tedious and error prone.  In the
\texttt{scripts/charonInterpreter/tools} directory, there is a script
called \texttt{xmlToLCM.py} which does this translation automatically.
This script will take a text file \texttt{xmlbjt.xml} which contains
the xml formatted parameter list seen above and create a new text file
called \texttt{xmlbjt.xml.LCMified} which contains the translated
interpreter formatted parameter list, viz
\begin{lstlisting}
Charon->Mesh,Source,string,Exodus File
Charon->Mesh->Exodus File,File Name,string,bjt2d_equ.exo
Charon->Mesh->Exodus File,Restart Index,int,3
\end{lstlisting}
These lines must be categorized as default, required and optional and
added to the parser input file appropriately.

The \texttt{Source} parameter in the first line,
\begin{lstlisting}
Charon->Mesh,Source,string,Exodus File
\end{lstlisting}
must always be present in an identical way.  It must not ever vary the
way Charon works now.  It is nothing but a useless burden to require a
user to put this into every input file they create.  This is an ideal
candidate for a default line.  The interpreter will always write it
into the parameter list and the user need never see it.  This is
specified in the parser input file as
\begin{lstlisting}
xmlDefault Charon->Mesh,Source,string,Exodus File
\end{lstlisting}
The name of the file the user wishes to import must always be
specified and is never a default name.  This is a good candidate for a
required input and is specified in the parser input file as
\begin{lstlisting}
xmlRequired Charon->Mesh->Exodus File,File Name,string,{filename}
\end{lstlisting}
where the filename variable replaces the explicit file name in the
example and ties this particular parameter to the interpreter input.
That is, \texttt{\{filename\}} gets replaced in the parameter list by
the string specified in the interpreter input file.

Lastly, we wish to include the restart index as an optional parameter.
Moreover, we may also wish for the restart index to take on a default
value if the option is not specified by the user.  The default value
for the restart index is specified in the same way the \texttt{Source}
parameter was specified,
\begin{lstlisting}
xmlDefault Charon->Mesh->Exodus File,Restart Index,int,-1
\end{lstlisting}
This creates a parameter \texttt{Restart Index} with a value of -1.
To create the optional parameter, the parser input syntax is similar
to default and required parameters, but must also contain the optional
keywords,
\begin{lstlisting}
xmlOptional (at index) Charon->Mesh->Exodus File,Restart Index,int,{index}
\end{lstlisting}
so the \texttt{Restart Index} appears twice in the parser input file.
In one it contains default value information and in the other,
optional user-supplied information.  Here again, the
\texttt{\{index\}} variable maps back to the interpreter input line.

By pulling all of this together, the entire parser input file now reads:
\begin{lstlisting}
interpreter name ImportStateFile

interpreter inputLine (Import State File) {filename} [(at Index) {index}]

interpreter shortHelp {Specify state file name and state index}

interpreter longHelp {Specify the name of the file (exodus) that contains the states for input/initial guess. <> Optionally, specify the index of the time plane or parameter plane if the file contains multiple states. <> filename is the name of the file.  It is case sensitive. <> index is the integer index of the state to be used as input.}

xmlRequired Charon->Mesh->Exodus File,File Name,string,{filename} 

xmlOptional (at Index) Charon->Mesh->Exodus File,Restart Index,int,{index}

xmldefault  Charon->Mesh,Source,string,Exodus File
xmldefault  Charon->Mesh->Exodus File,Restart Index,int,-1
\end{lstlisting}
The parser is then generated by executing the generateInterpreter.py
script.

The final step the developer must execute is to validate the parser by
running the charon interpreter on a file that contains the new
interpreter line.  Say our file is \texttt{bjt\_2d.inp} and has the
line
\begin{lstlisting}
import state file bjt2d_equ.exo at index 3
\end{lstlisting}
Executing
\begin{lstlisting}
charonInterpreter.py --input bjt_2d.inp
\end{lstlisting}
produces a file named \texttt{bjt\_2d.inp.xml} with the contents
\begin{lstlisting}
<ParameterList name="Charon" >
  <ParameterList name="Mesh" >
    <ParameterList name="Exodus File" >
      <Parameter name="Restart Index" type="int" value="3" />
      <Parameter name="File Name" type="string" value="bjt2d_equ.exo" />
    </ParameterList>
    <Parameter name="Source" type="string" value="Exodus File" />
  </ParameterList>
</ParameterList>
\end{lstlisting}
Other than ordering of lines, this parameter list is identical.  In
this simple example, this is easy to see. Once again, however, if this
parameter list were long and complicated, the validation will be
tedious and error prone.  Another script in the tools directory will
make the comparison and provide a kind of diff between the two files.
It reports what exists in one file that does not exist in the other.
In this case, the script \texttt{compareParameterLists.py} can be used
to compare \texttt{bjt\_2d.inp.xml} and the file that contained the
original xml input we worked from to get the following report:
\begin{lstlisting}
compareParameterLists.py bjt_2dOriginal.xml bjt_2d.inp.xml
The following is a list of items contained in bjt_2dOriginal.xml, but not in bjt_2d.inp.xml



The following is a list of items contained in bjt_2d.inp.xml, but not in bjt_2dOriginal.xml

\end{lstlisting}
If something went wrong with the mapping in the interpreter and the
xml generated turned out to be
\begin{lstlisting}
<ParameterList name="Charon" >
  <ParameterList name="Mesh" >
    <ParameterList name="Exodus File" >
      <Parameter name="Restart Index" type="int" value="300" />
      <Parameter name="File Name" type="string" value="bjt2d_equ.exo" />
    </ParameterList>
    <Parameter name="Source" type="string" value="Exodus File" />
  </ParameterList>
</ParameterList>
\end{lstlisting}
the report would be
\begin{lstlisting}
The following is a list of items contained in bjt_2dOriginal.xml, but not in bjt_2d.inp.xml

Charon->Mesh->Exodus File,Restart Index,int,3


The following is a list of items contained in bjt_2d.inp.xml, but not in bjt_2dOriginal.xml

Charon->Mesh->Exodus File,Restart Index,int,300
\end{lstlisting}
which highlights the difference.  A more complicated example would be
if the nesting were incorrect.  The compare script is too naive to
figure out how nesting should have been done.  When it finds nested
parameters in one file and not in another, it stops ever reporting the
one contains a parameter list not contained in the other.  For
example, suppose instead of the parameter list \texttt{Exodus File},
the parser input file contained a typo, \texttt{Eoxdus File}.  The
compare script is not smart enough to figure out the typo, but will
indicate that the two files differ with the following report:
\begin{lstlisting}
The following is a list of items contained in bjt_2dOriginal.xml, but not in bjt_2d.inp.xml

Charon->Mesh->Exodus File


The following is a list of items contained in bjt_2d.inp.xml, but not in bjt_2dOriginal.xml

Charon->Mesh->Eoxdus File
\end{lstlisting}
Nothing below the level of \texttt{Exodus File} is reported, but
sufficient information is provided to the developer to indicate the
error.

\subsection{Creating a Block Parser}

The line parsers in the Charon interpreter are where the bulk of the
work is done in parsing and running a Charon job.  As was described in
the previous section, a single line of interpreter input can map to
many lines of parameter lists.  It still often makes sense to bundle
line parsers into sensible groups.  This makes the input file neater,
more organized and easier to navigate.  These groups in the
interpreter are called blocks and they have their own parsers.  Block
parsers are in principle simpler than line parsers.  Their sole
purpose for existence is for organization and to make data input
simpler.  A good example of a block in the Charon input is where
material models are spelled out for a common region of a device, e.g.
\begin{lstlisting}
start Material Block siliconParameter
      material name is Silicon
      relative permittivity = 11.9
      start step junction doping
      	    acceptor concentration = 1e16
	    donor concentration = 1e16
	    junction location = 0.5
	    dopant order is PN
      end step junction doping
end Material Block siliconParameter
\end{lstlisting}
In this instance, two blocks are nested.  There is no limitation to
the depth that blocks may be nested, but it should be limited to as
few layers as makes sense to keep things simple.  Block parsers
control this nesting in an organized way and provide some of the
mapping information.  

Every line inside the blocks that begins neither with ``start'' nor
``end'' is just a line parser like the one created in the previous
section's tutorial.  The way the parser input files and thus the
parsers are organized mirrors the nested structure of the blocks.  So
the root directory for the parser input files,
\texttt{charonInterpreter/parseGenerator/parseInputs} contains the
block parser for the material block.  The structure and composition of
the block parser input file is similar though simpler than the line
parser input files.  The name of the block parser input file is
required only to have the suffix \texttt{blockinp}, but the following
convention has been used exclusively,
\begin{lstlisting}
create<BlockParserName>.blockinp
\end{lstlisting}
In this example, \texttt{BlockParserName} = \texttt{MaterialBlock} and
the first line in the block parser makes that specification,
\begin{lstlisting}
interpreterBlock name MaterialBlock
\end{lstlisting}
And just as in the line parser input file, there must be a line which
specifies the keywords that trigger the block parser,
\begin{lstlisting}
interpreterBlock (start Material Block) {MaterialBlockName}
\end{lstlisting}
where as before the key words are enclosed in parentheses and the
variable MaterialBlockName is enclosed in curly braces.  This
represents the minimum input to create a block parser.  It is often
the only input.

With blocks, there is a directory structure that accompanies and
mirrors the block structure.  Under the \texttt{parseInputs}
directory, the developer must create a subdirectory that has the same
name as the block parser.  In this example, it is the developer's
burden to create the\texttt{parseInputs/MaterialBlock} subdirectory.
Every line parser that accompanies every line in the block has an
input file that lives in this subdirectory.  That is, the line parsers
for
\begin{lstlisting}
  material name is Silicon
\end{lstlisting}
and 
\begin{lstlisting}
  relative permittivity = 11.9
\end{lstlisting}
namely \texttt{createMaterialName.inp} and
\texttt{createRelativePermattivity.inp} are located in the
\texttt{parseInputs/MaterialBlock} subdirectory.

This example contains one nested block layer.  The
directory-subdirectory structure simply repeats itself with nested
blocks.  There is a \texttt{createStepDoping.blockinp} block parser
input file and a \texttt{StepDoping} subdirectory in the
\texttt{MaterialBlock} block directory.  All line parse input files
related to the step doping block are located in the
\texttt{StepDoping} directory.

It is important to note that very commonly, parameters created by line
parsers in blocks almost always need to know the block name in order
to get the parameter list correct.  Every line parser inside the
blocks will have defined any variable specified in the block parser.
In this example, the variable \texttt{MaterialBlockName} is defined
and can be used in every line parser input file that exists under the
\texttt{MaterialBlock} directory tree.  The \texttt{MaterialName} line
parser input file is
\begin{lstlisting}
interpreter name MaterialName

interpreter inputLine (material name) is {materialName}

interpreter shortHelp {Set the Material name }

interpreter longHelp {Set the Material name <> materialName is the material in this material block (e.g. Silicon)}

xmlRequired Charon->Closure Models->{MaterialBlockName},Material Name,string,{materialName}
\end{lstlisting}

\subsection{Custom Parameter Priority}

It can happen on occasion that a parameter gets defined more than
once.  Most of the time, this presents no problem.  But sometimes it
does.  An example of this is doing potential sweeps on contacts.  An
interpreter input line for sweeping is
\begin{lstlisting}
BC is ohmic for anode on silicon swept from 0.5 to 0.25
\end{lstlisting}
The parser that processes this line will create the appropriate
parameter list for the boundary condition.  It also creates as
xmlOptional parameters the necessary inputs to switch the run from NOX
solver to LOCA solver and sets LOCA stepper information.  This parser
will set the initial step size of the sweep to 1.0.  A step size of 1
volt is often far too large to take and so a smaller step size should
be specified.  This can be done in an optional block for sweep
parameters,
\begin{lstlisting}
start sweep options
      initial step size = 0.25
end sweep options
\end{lstlisting}
With this, a parameter has been set that creates an initial step size
of 1.0 and a second one has been set that creates an initial step size
of 0.25.  This presents a potential problem because in the first
instance, the parameter is xmlOptional and the second, the parameter
is xmlRequired.  Optionals and Requireds have the same priority and
the one which will get used is the one last encountered in the list.
This makes the input file order dependent and undesirable.  This is
remedied by adding a custom priority tag to the desired specification.
When the user adds the optional sweep options block, anything
specified in there must take priority over anything else that is
default behavior.  The tag is set in the line parser input file for
the initial step size in the sweep option block as follows:
\begin{lstlisting}
interpreter name InitialStepSize

interpreter inputLine (Initial Step Size) = {stepSize}

interpreter shortHelp {Specify initial step size for parameter sweep}

interpreter longHelp {Specify initial step size for parameter sweep <> {stepSize} = the step size}

xmlRequired Charon->Solution Control->LOCA->Step Size,Initial Step Size,double,{stepSize} priority 5
\end{lstlisting}
If the priority tag is missing from the xmlRequired line, default
priority is assigned.  If the tag is present, the requested priority
is assigned.


\subsection{Anonymous Parameter Lists}

There are unfortunate instances in the Charon parameter lists that a
parameter list is unnamed.  This is called an anonymous parameter list
and it does create an organizational hazard.  The interpreter relies
on a unique nesting path to do its mapping from interpreter
specifications into parameter lists.  These are managed in a special
way in the interpreter and some care must be taken to get them right.
The parameter list name \texttt{ANONYMOUS} is always used where there
is an unnamed parameter list.  If the unnamed parameter list has no
peers, that is if it sits alone at its particular nesting level and
path, \texttt{ANONYMOUS} is all that is required to handle the mapping
correctly.  However, if there is more than one, \texttt{ANONYMOUS} is
not unique and the mapping will not be done correctly.  An important
example of this is contact boundary conditions.  The boundary
conditions on a diode are as follows:
\begin{lstlisting}
<ParameterList name="Charon">
    <ParameterList name="Boundary Conditions">
        <ParameterList>
            <Parameter name="Type" type="string" value="Dirichlet"/> 
            <Parameter name="Sideset ID" type="string" value="anode"/> 
            <Parameter name="Element Block ID" type="string" value="silicon"/> 
            <Parameter name="Equation Set Name" type="string" value="ALL_DOFS"/> 
            <Parameter name="Strategy" type="string" value="Ohmic Contact"/> 
            <ParameterList name="Data">
                <Parameter name="Voltage" type="double" value="0.5"/>
            </ParameterList>
        </ParameterList>    
        <ParameterList>
            <Parameter name="Type" type="string" value="Dirichlet"/> 
            <Parameter name="Sideset ID" type="string" value="cathode"/> 
            <Parameter name="Element Block ID" type="string" value="silicon"/> 
            <Parameter name="Equation Set Name" type="string" value="ALL_DOFS"/> 
            <Parameter name="Strategy" type="string" value="Ohmic Contact"/>
            <ParameterList name="Data">
                <Parameter name="Voltage" type="double" value="0"/>
            </ParameterList>
        </ParameterList>
    </ParameterList>
</ParameterList>
\end{lstlisting}
The anode and cathode are peer parameter lists and are unnamed and are
not unique.  Internally, Trilinos-Teuchos will replace anonymous
parameter lists with assigned names \texttt{child0}, \texttt{child1}
etc., but the user never sees it.  The interpreter can only handle
parameters with a unique path.  The way the interpreter manages this
is that any parameter list named \texttt{ANONYMOUS} will get written
to the mapped parameter list as unnamed.  This is true even if
\texttt{ANONYMOUS} represents only a fraction of the name.  The parser
input file in part for boundary conditions is
\begin{lstlisting}
interpreter name OhmicBC

interpreter inputLine (BC is ohmic for) {sidesetID} on {geometryBlock} [(fixed at) {potential}[ (swept from) {potential1} to {potential2}]]

interpreter shortHelp {Specify the potential on a contact}

interpreter longHelp {Specify the potential on a contact. <> sidesetID is the contact name/type <> geometryBlock is the geometry name the contact is attached to <> potential is the value in volts}

xmlRequired Charon->Boundary Conditions->{sidesetID}ANONYMOUS,Type,string,Dirichlet
xmlRequired Charon->Boundary Conditions->{sidesetID}ANONYMOUS,Sideset ID,string,{sidesetID}
xmlRequired Charon->Boundary Conditions->{sidesetID}ANONYMOUS,Element Block ID,string,{geometryBlock}
xmlRequired Charon->Boundary Conditions->{sidesetID}ANONYMOUS,Equation Set Name,string,ALL_DOFS
xmlRequired Charon->Boundary Conditions->{sidesetID}ANONYMOUS,Strategy,string,Ohmic Contact
\end{lstlisting}
Because the sidesetID that must be specified for the boundary
condition will be unique for that parameter list, it is prepended to
the ANONYMOUS part of the parameter.  This has the effect of giving
the interpreter a unique name to work with.  When the parameter list
is ultimately mapped and written, the entire
\texttt{{sidesetID}ANONYMOUS} string gets replaced with nothing and
becomes and unnamed parameter list.


